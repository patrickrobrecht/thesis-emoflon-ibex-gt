% !TeX spellcheck = en_US
\chapter{Goals}
\label{goals}
The main goal of the thesis is to implement a graph transformation tool with support for the features described in Section~\ref{requirements}.
The most important objectives are the integration with a TGG tool based on incremental pattern matching and with a GPL (Java).

\section{Research Questions}
\label{research-questions}
The enumeration below lists the research questions that will be investigated in the thesis.
\begin{description}
	\item[RQ 1]
		Which tasks are best solved using an incremental pattern matcher?
	\item[RQ 2]
		How can a GT and TGG tool be seamlessly integrated for developers and end users?
	\item[RQ 3]
		How can we integrate graph transformations seamlessly into Java code using the power of Java 8 features such as streams\footnote{\url{http://www.oracle.com/technetwork/articles/java/ma14-java-se-8-streams-2177646.html}}?
\end{description}

\section{Tasks}
\label{tasks}
In this section we will give an overview over the tasks that need to be performed for the implementation.
During the incremental implementation and testing process the following features shall be realized (ordered by importance):
\begin{enumerate}
	\item Simple constraints (just black nodes and references)
	\item Creation and deletion of nodes and references (green and red elements)
		\begin{enumerate}
			\item Pushout semantics (SPO as default, DPO)
			\item Parameters for context and deleted nodes
		\end{enumerate}
	\item Rule refinement (a rule may refine multiple rules)
	\item Attribute conditions and assignments
	\item Negative application conditions (or more complex graph conditions)
\end{enumerate}
The architecture of the planned graph transformation tool is illustrated in Figure \ref{fig:model-transformations}. 
The green part is the Xtext\footnote{\url{https://www.eclipse.org/Xtext/}} based text editor for GT rules.
The blue one deals with the internal representation of graph transformation rules and the pattern abstraction (independent from a concrete pattern matching engine),
while the orange one is the integration into Democles as the current pattern matcher used by eMoflon::IBeX.
If one wants to use eMoflon::IBeX with another pattern matching engine, just the orange part needs to be replaced with an implementation for the other engine.  

\begin{figure}[h!]
	\centering
	% !TeX encoding = UTF-8
% !TeX spellcheck = en_US

\begin{tikzpicture} [
		auto,
		node distance = 2.25cm,
		node/.style = {
			rectangle,
			rounded corners,
			thick,
			text width = 6em,
			text centered,
			minimum height = 3em,
		},
		line/.style = {
			draw,
			thick,
			->,
			shorten >=2pt,
		},
		line-caption/.style = {
			font = \footnotesize
		},
		ibex-ui/.style = {
			draw = green!60!lime,
			fill = green!60!lime,
		},
		ibex/.style = {
			draw = blue!20!cyan,
			fill = blue!20!cyan,
		},
		ibex-democles/.style = {
			draw = red!10!orange,
			fill = red!10!orange
		}
	]

	\node[node, ibex-ui] (f) {
		\texttt{gt} text file
	};

	\node[node, ibex-ui, below of = f] (e) {
		Editor Model
	};
	\node[node, ibex-ui, left = 2cm of e] (ev) {
		Editor Model Visualization
	};
	\node[node, ibex-ui, right = 2cm of e] (em) {
		Xtext-based Meta-Model
	};

	\node[node, ibex, below of = e, xshift = 1.6cm] (g) {
		GT API Model
	};
	\node[node, ibex, right = 2cm of g] (gm) {
		GT API Meta-Model
	};
	\node[node, ibex, below of = g] (gc) {
		Java API
	};

	\node[node, ibex, below of = e, xshift = -1.6cm] (i) {
		IBeX Patterns
	};
	\node[node, ibex, left = 2cm of i] (im) {
		IBeX Pattern Meta-Model
	};
	\node[node, ibex, below of = i] (iv) {
		IBeX Pattern Visualization
	};

	\node[node, ibex-democles, below of = im] (d) {
		Democles Patterns
	};
	\node[node, ibex-democles, below of = d] (dm) {
		Democles Meta-Model
	};
	\node[node, ibex-democles, below of = iv] (dv) {
		Democles Visualization
	};

	\begin{scope} [
			every path/.style = line, ibex-ui,
			every node/.style = line-caption
		]
		\path (f) -- node[right] {Xtext transformation} (e);
		\path (e) -- node[below] {visualized in} (ev);
		\path (e) -- node[below] {conforms to} (em);
	\end{scope}

	\begin{scope} [
			every path/.style = line, ibex,
			every node/.style = line-caption
		]
		\path (e) -- node[left, xshift = -0.2cm] {transformation} (i.north);
		\path (e) -- node[right, xshift = 0.2cm] {transformation} (g.north);

		\path (i) -- node[below] {conforms to} (im);
		\path (i) -- node[right] {visualized in} (iv);

		\path (g) -- node[below] {conforms to} (gm);
		\path (g) -- node[right] {code generation} (gc);
	\end{scope}

	\begin{scope} [
			every path/.style = line, ibex-democles,
			every node/.style = line-caption
		]
		\path (i) -- node[left, yshift = 0.2cm] {transformation} (d.east);
		\path (d) -- node[right] {conforms to} (dm);
		\path (d) -- node[right, xshift = 0.2cm] {visualized in} (dv);
	\end{scope}
\end{tikzpicture}

	\caption{Model transformations for eMoflon::IBeX-GT}
	\label{fig:model-transformations}
\end{figure}

To implement the features listed above, the following steps have to be completed for each feature:
\begin{enumerate}
	\item Define a textual concrete syntax and provide an editor.
	\item Provide a visualization of the editor model\footnote{\eg using PlantUML}.
	\item Define/extend an internal GT meta-model.
	\item Transform the editor model to an internal GT model.
	\item Transform the internal GT model to an IBeX pattern.\footnote{The IBeX pattern shall abstract from GT such that it can be shared with the TGG part of eMoflon::IBeX.
	In addition the IBeX pattern abstracts from different potential variants of converting a GT rule into patterns.}
	\item Provide a graphical visualization of the IBeX pattern.
	\item Transform the IBeX pattern to a Democles pattern.
	\item Design/extend the Java API.
	\item Write JUnit tests for the API using patterns/rules with the feature.
	\item Write end-user documentation (\ie add a feature description and an example in the handbook).
	\item Deploy the feature with the next eMoflon::IBeX version (for end-user testing).
\end{enumerate}
