% !TeX encoding = UTF-8
% !TeX spellcheck = en_US

\chapter{Related Work}
\label{related-work}
This chapter presents existing graph transformation tools.
Section~\ref{tools} analyzes them with respect to our requirements from Chapter~\ref{requirements}, while Section~\ref{comparison} summarizes the comparison results.

\section{Graph Transformation Tools}
\label{tools}

There are already some tools available supporting the modification of attributed typed graphs via graph transformation.
Many of them are used for academic purposes, while others are also applied in industry.

\textit{AGG} (``The Attributed Graph Grammar System'') \cite{AGGWebsite}, developed by Olga Runge (TU Berlin), interprets graph transformation rules with NACs and control structures based on the SPO semantics.
While the latest version was released in 2017, the manual \cite{AGGManual} is outdated (already ten years old).
AGG offers an API for integration with Java programs, for which detailed documentation could not be found.
The examples available on the website indicate that the API is not typed, thus requires casting.

\textit{AToMPM} (``A Tool for Multi-formalism and Meta-Modelling'') \cite{AToMPMDocs} is an interpreter and code generator for model queries and unidirectional transformations (both endogenous and exogenous) with a web-based user interface.
The development of the tool is a joint project of universities in Montreal, Antwerp and Alabama.
An integration into a general purpose language is not directly supported.
AGG and AtoMPM do not support modularity for rules and parameterized rules.

\textit{eMoflon} \cite{eMoflonWebsite} is a GT and TGG tool developed at TU Darmstadt and Paderborn University.
Only eMoflon::SDM (the GT part) is checked for compliance with our requirements.
The EMF-based Eclipse plugin relies on code generation.
An end-user handbook \cite{eMoflonHandbook} is available, introducing eMoflon based on examples.

\textit{EMorF} \cite{EMorFWebsite}, developed by Solunar GmbH, is an EMF-based Eclipse plugin with support for model queries and model transformations (model modification as well as bidirectional transformations with TGGs).
It offers an API for integration into Java programs, but no detailed API documentation is available.
The parameters of the API are not typed; the same holds for objects in matches so that casts are required.
The rule specification is just introduced based on one example.
The last release 0.4.2 was published in 2012, since then, the development has been discontinued.

\textit{GRAPE} (``Graph Rewriting And Persistence Engine'') \cite{GRAPE}, a GT interpreter by Jens Weber, can be integrated into Clojure programs.
The GRAPE interpreter uses SPO semantics by default, but rules may specifiy that they are applied according to DPO semantics.
Rules can be parameterized, but cannot refine other rules.
Documentation of the textual syntax \cite{GRAPEWebsite} with many examples is available online.

\textit{GrGen.NET} (``Graph Rewrite Generator'') \cite{GrGenWebsite} offers declarative rule specification for transformations, applied by a code generator.
Rules may share subpatterns, but cannot refine other rules.
Their application follows SPO semantics.
A type-safe API enables integration into C\# code.
Detailed documentation \cite{GrGenUserManual} is available.

\textit{GROOVE} (``GRaphs for Object-Oriented Verification'') \cite{GrooveWebsite} provides graph transformations based on SPO rules.
Graphs can be queried with Prolog, but graphs cannot be transformed via a general purpose language.
Groove also lacks a modularity concept in the rule specification language.

\textit{Henshin EMF} \cite{HenshinEMFWebsite} is an Eclipse-based graph transformaton tool providing an interpreter API for the usage in Java code.
When using the API, rules must be loaded into a so-called unit which can be applied to the model.\footnote{\url{https://wiki.eclipse.org/Henshin/Interpreter}}
The interaction with the API classes relies on rule and node names.
The interface is not typed such that casting from \texttt{Object} to the concrete type is necessary.
The interpreter supports both SPO and DPO which can be arbitrarily mixed.
The tool is documented in a wiki \cite{HenshinEMFWiki} accompanied by examples.

\textit{VIATRA} \cite{VIATRAWebsite}, part of the Eclipse Modeling Project, is based on an incremental query backend and focuses on pattern matching and not on graph transformation.
A Java API is available for integration into Java applications.
A typed interface is generated for the API.\footnote{\url{https://www.eclipse.org/viatra/documentation/query-api.html}}
Even if the focus is not on algebraic graph transformation, VIATRA offers major support for many of our requirements, but lacks support for rule applications and integration with a TGG tool.
From the tools presented in this chapter, it is the only one that supports incrementality.
VIATRA offers composable and reusable patterns, but there is no modularity concept on rule level.
No visualization is provided in the textual editor.
The website offers tutorials and documentation.

\section{Comparison of Existing Graph Transformation Tools}
\label{comparison}
The comparison of the tools introduced in the previous section is summarized in Figure~\ref{fig-comparison}.\footnote{Horizontal and vertical lines in the table are just for easier reading, the aspects are ordered as in the previous section.}
This chapter focuses on tools with at least one release since 2015 or special support for one of the requirements.
It is partly based on the comparison of model transformation tools published by Kahani and Cordy \cite{ComparisonAndEvaluationOfModelTransformationTools, ComparisonOfModelTransformationToolsWebsite}.
Please note that the comparison takes solely our requirements into consideration and ignores other aspects completely.
The tools have been analyzed in the version stated in Figure~\ref{fig-comparison} according to their official documentation.

The symbol \yes~means that the requirement is fully fulfilled and \no~states that the requirement is not fulfilled.
The notation (\yes) indicates that the requirement is only partially fulfilled, as stated in the previous section or in the footnotes.

VIATRA is the only tool with support for incrementality, while only eMoflon and eMorF combine GT and TGG within one tool.
Most tools support integration into a general purpose language, but not all APIs are typed and well documented (see previous section).

No tool supports modularity in the sense of pattern refinement, only GrGen.NET and VIATRA allow to share the specification of graph structures between patterns.
Only GrGen.NET and Henshin EMF support application conditions which are more complex than NACs.
Except AtoMPM, all tools support one of the common modeling standards EMF (Eclipse Modeling Framework), GXL (Graph eXchange Language), or Neo4J (a graph database management system).

\newcommand{\rotateText}[1]{\rotatebox[origin=cb]{75}{{#1}}}
\setlength{\tabcolsep}{4pt}
\begin{longtable}{p{41mm}|ccc|ccc|ccc}
	\toprule
	Tool, Version $\rightarrow$ \newline ~ \newline Feature $\downarrow$
	& \rotateText{AGG} 
	& \rotateText{AtoMPM} 
	& \rotateText{eMoflon::SDM}
	& \rotateText{EMorF}
	& \rotateText{GRAPE}
	& \rotateText{GrGen.NET} 
	& \rotateText{GROOVE}
	& \rotateText{Henshin EMF} 
	& \rotateText{VIATRA} \\
	& 2.1
	& 0.6.1
	& 2.32.0
	& 0.4.2
	& 0.1.1
	& 4.5.2
	& 5.7.2
	& 1.4.0
	& 1.6.1 \\
	\midrule
	Incrementality
	& \no
	& \no
	& \no
	& \no
	& \no
	& \no
	& \no
	& \no
	& \yes \\
	Interpreter
	& \yes
	& \yes
	& \no
	& \yes 
	& \yes
	& \no 
	& \yes 
	& \yes
	& \yes \\
	Integration with a TGG tool
	& \no
	& \no
	& \yes
	& \yes
	& \no
	& \no
	& \no
	& \no
	& \no \\
	\midrule
	Mature integration with a general purpose language\footnote{\yes indicates that an API for integration into Java, C\#, or Clojure is available.
	GROOVE allows model queries via Prolog.}
	& \yes
	& \no
	& \no
	& \yes
	& \yes
	& \yes
	& (\yes)
	& \yes
	& \yes \\
	Dedicated support for model queries
	& \no
	& \yes
	& \no
	& \yes
	& \no
	& \no
	& \no
	& \no
	& \yes \\
	DPO or SPO semantics
	& SPO
	& ?\footnote{The documentation does not describe the pushout semantics.}
	& SPO
	& SPO
	& both
	& SPO
	& both
	& both
	& -- \\
	\midrule
	Modularity on rule level
	& \no
	& \no
	& \no
	& \no
	& \no
	& (\yes)
	& \no
	& \no
	& (\yes) \\
	Application conditions\footnote{For application conditions, a \yes indicates support for more complex application conditions than NACs.}
	& NACs
	& NACs
	& NACs
	& OCL\footnote{EMorF one can use constraints defined in the Object Constraint Language (OCL) to restrict the applicability of a rule.}
	& NACs
	& \yes
	& NACs
	& \yes
	& NACs \\
	Attribute manipulation\footnote{Most tools only support a limited set of attribute conditions, indicated by (\yes).
		A \yes indicates that the user can define custom attribute relations.}
	& (\yes)
	& (\yes)
	& \yes
	& (\yes)
	& (\yes)
	& \yes
	& (\yes)
	& (\yes)
	& (\yes) \\
	\midrule
	Textual concrete syntax and visualization\footnote{AGG, AtoMPM, EMorF, eMoflon::SDM, GROOVE, and Henshin EMF come with a graphical editor,
		while GRAPE and GrGen.NET provide a textual editor and a generated visualization.
		VIATRA makes use of a textual syntax, but does not provide any visualization.}
	& \no
	& \no
	& \no
	& \no
	& \yes
	& \yes
	& \no
	& \no
	& \no \\
	Modeling standard
	& GXL
	& \no
	& EMF
	& EMF
	& Neo4J
	& EMF
	& GXL
	& EMF
	& EMF \\
	End-user documentation
	& (\yes)
	& \yes
	& \yes
	& \no
	& \yes
	& \yes
	& \yes
	& \yes
	& \yes \\
	\bottomrule 
	\caption{Comparison of Graph Transformation Tools}
	\label{fig-comparison}
\end{longtable}
